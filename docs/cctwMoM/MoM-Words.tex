  We similarly define $\hEndMin_{\EndStep}$ as `minimal human wealth,' the
  present discounted value of labor income if the shocks were to take on
  their worst possible value in every future period \PermShkOn
  {$\TranShkEmp_{t+n} = \TranShkEmpMin ~\forall~n>0$ and $\PermShk_{t+n} =
    \PermShkMin ~\forall~n>0$} {$\TranShkEmp_{t+n} = \TranShkEmpMin
    ~\forall~n>0$} (which we define as corresponding to the beliefs of a
  `pessimist').

  \ctw{}{We will call a `realist' the consumer who correctly perceives the true
    probabilities of the future risks and optimizes accordingly.}

  A first useful point is that, for the realist, a lower bound for the
  level of market resources is $\ushort{m}_{t} = -\hEndMin_{\EndStep}$, because
  if ${m}_{t}$ equalled this value then there would be a positive finite
  chance (however small) of receiving \PermShkOn
  {$\TranShkEmp_{t+n}=\TranShkEmpMin$ and $\PermShk_{t+n}=\PermShkMin$}
  {$\TranShkEmp_{t+n}=\TranShkEmpMin$}
  in
  every future period, which would require the consumer to set ${c}_{t}$
  to zero in order to guarantee that the intertemporal budget constraint
  holds\ctw{.}{~(this is the multiperiod generalization of the discussion in
    section \ref{subsec:LiqConstrSelfImposed} explaining the derivation of the `natural borrowing constraint' for period $T-1$,
    $\ushort{a}_{T-1}$).}  Since consumption of zero yields negative
  infinite utility, the solution to realist consumer's problem is not well
  defined for values of ${m}_{t} < \ushort{m}_{t}$, and the limiting
  value of the realist's ${c}_t$ is zero as ${m}_{t} \downarrow \ushort{m}_{t}$.

  Given this result, it will be convenient to define `excess' market
  resources as the amount by which actual resources exceed the lower
  bound, and `excess' human wealth as the amount by which mean expected human wealth
  exceeds guaranteed minimum human wealth:
  \begin{equation*}\begin{gathered}\begin{aligned}
        \aboveMin \mNrm_{t}  & = {m}_{t}+\overbrace{\hEndMin_{\EndStep}}^{=-\ushort{m}_{t}}
        \\  \aboveMin \hNrm_{\EndStep}  & = \hNrm_{\EndStep}-\hEndMin_{\EndStep}.
      \end{aligned}\end{gathered}\end{equation*}

  We can now transparently define the optimal
  consumption rules for the two perfect foresight problems, those of the
  `optimist' and the `pessimist.'  The `pessimist' perceives human
  wealth to be equal to its minimum feasible value $\hEndMin_{\EndStep}$ with certainty, so
  consumption is given by the perfect foresight solution
  \begin{equation*}\begin{gathered}\begin{aligned}
        \cFuncBelow_{t}(m_{t})  & = ({m}_{t}+\hEndMin_{\EndStep})\MPCmin_{t}
        \\  & = \aboveMin \mNrm_{t}\MPCmin_{t}
        .
      \end{aligned}\end{gathered}\end{equation*}

  The `optimist,' on the other hand, pretends that there is no uncertainty
  about future income, and therefore consumes
  \begin{equation*}\begin{gathered}\begin{aligned}
        \cFuncAbove_{t}(m_{t})  & = ({m}_{t} +\hEndMin_{\EndStep} - \hEndMin_{\EndStep} + \hNrm_{\EndStep} )\MPCmin_{t}
        \\    & = (\aboveMin \mNrm_{t} + \aboveMin \hNrm_{\EndStep})\MPCmin_{t}
        \\      & = \cFuncBelow_{t}(m_{t})+\aboveMin \hNrm_{\EndStep} \MPCmin_{t}
        .
      \end{aligned}\end{gathered}\end{equation*}

  It seems obvious that the spending of the realist will be strictly greater
  than that of the pessimist and strictly less than that of the
  optimist.  Figure~\ref{fig:IntExpFOCInvPesReaOptNeedHiPlot} illustrates the proposition for the consumption rule in period $T-1$.
