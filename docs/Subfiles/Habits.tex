% -*- mode: LaTeX; TeX-PDF-mode: t; -*- # Tell emacs the file type (for syntax)
% -*- mode: LaTeX; TeX-PDF-mode: t; -*- # Tell emacs the file type (for syntax)
% LaTeX path to the root directory of the current project, from the directory in which this file resides
% and path to econtexPaths which defines the rest of the paths like \FigDir
\providecommand{\econtexRoot}{}\renewcommand{\econtexRoot}{.}
\providecommand{\econtexPaths}{}\renewcommand{\econtexPaths}{\econtexRoot/Resources/econtexPaths}
% -*- mode: LaTeX; TeX-PDF-mode: t; -*- 
% The \commands below are required to allow sharing of the same base code via Github between TeXLive on a local machine and Overleaf (which is a proxy for "a standard distribution of LaTeX").  This is an ugly solution to the requirement that custom LaTeX packages be accessible, and that Overleaf prohibits symbolic links
\providecommand{\packages}{\econtexRoot/Resources/texmf-local/tex/latex}
\providecommand{\econtex}{\packages/econtex}
\providecommand{\econark}{\econtexRoot/Resources/texmf-local/tex/latex/econark}
\providecommand{\econtexSetup}{\econtexRoot/Resources/texmf-local/tex/latex/econtexSetup}
\providecommand{\econtexShortcuts}{\econtexRoot/Resources/texmf-local/tex/latex/econtexShortcuts}
\providecommand{\econtexBibMake}{\econtexRoot/Resources/texmf-local/tex/latex/econtexBibMake}
\providecommand{\econtexBibStyle}{\econtexRoot/Resources/texmf-local/bibtex/bst/econtex}
\providecommand{\econtexBib}{economics}
\providecommand{\notes}{\econtexRoot/Resources/texmf-local/tex/latex/handout}
\providecommand{\handoutSetup}{\econtexRoot/Resources/texmf-local/tex/latex/handoutSetup}
\providecommand{\handoutShortcuts}{\econtexRoot/Resources/texmf-local/tex/latex/handoutShortcuts}
\providecommand{\handoutBibMake}{\econtexRoot/Resources/texmf-local/tex/latex/handoutBibMake}
\providecommand{\handoutBibStyle}{\econtexRoot/Resources/texmf-local/bibtex/bst/handout}

\providecommand{\FigDir}{\econtexRoot/Figures}
\providecommand{\CodeDir}{\econtexRoot/Code}
\providecommand{\DataDir}{\econtexRoot/Data}
\providecommand{\SlideDir}{\econtexRoot/Slides}
\providecommand{\TableDir}{\econtexRoot/Tables}
\providecommand{\ApndxDir}{\econtexRoot/Appendices}

\providecommand{\ResourcesDir}{\econtexRoot/Resources}
\providecommand{\rootFromOut}{..} % APFach back to root directory from output-directory
\providecommand{\LaTeXGenerated}{\econtexRoot/LaTeX} % Put generated files in subdirectory
\providecommand{\econtexPaths}{\econtexRoot/Resources/econtexPaths}
\providecommand{\LaTeXInputs}{\econtexRoot/Resources/LaTeXInputs}
\providecommand{\LtxDir}{LaTeX/}
\providecommand{\EqDir}{\econtexRoot/Equations} % Put generated files in subdirectory

\providecommand{\local}{\LaTeXInputs/local}

\documentclass[\econtexRoot/SolvingMicroDSOPs]{subfiles}
% -*- mode: LaTeX; TeX-PDF-mode: t; -*- # Tell emacs the file type (for syntax)
% LaTeX path to the root directory of the current project, from the directory in which this file resides
% and path to econtexPaths which defines the rest of the paths like \FigDir
\providecommand{\econtexRoot}{}\renewcommand{\econtexRoot}{.}
\providecommand{\econtexPaths}{}\renewcommand{\econtexPaths}{\econtexRoot/Resources/econtexPaths}
% -*- mode: LaTeX; TeX-PDF-mode: t; -*- 
% The \commands below are required to allow sharing of the same base code via Github between TeXLive on a local machine and Overleaf (which is a proxy for "a standard distribution of LaTeX").  This is an ugly solution to the requirement that custom LaTeX packages be accessible, and that Overleaf prohibits symbolic links
\providecommand{\packages}{\econtexRoot/Resources/texmf-local/tex/latex}
\providecommand{\econtex}{\packages/econtex}
\providecommand{\econark}{\econtexRoot/Resources/texmf-local/tex/latex/econark}
\providecommand{\econtexSetup}{\econtexRoot/Resources/texmf-local/tex/latex/econtexSetup}
\providecommand{\econtexShortcuts}{\econtexRoot/Resources/texmf-local/tex/latex/econtexShortcuts}
\providecommand{\econtexBibMake}{\econtexRoot/Resources/texmf-local/tex/latex/econtexBibMake}
\providecommand{\econtexBibStyle}{\econtexRoot/Resources/texmf-local/bibtex/bst/econtex}
\providecommand{\econtexBib}{economics}
\providecommand{\notes}{\econtexRoot/Resources/texmf-local/tex/latex/handout}
\providecommand{\handoutSetup}{\econtexRoot/Resources/texmf-local/tex/latex/handoutSetup}
\providecommand{\handoutShortcuts}{\econtexRoot/Resources/texmf-local/tex/latex/handoutShortcuts}
\providecommand{\handoutBibMake}{\econtexRoot/Resources/texmf-local/tex/latex/handoutBibMake}
\providecommand{\handoutBibStyle}{\econtexRoot/Resources/texmf-local/bibtex/bst/handout}

\providecommand{\FigDir}{\econtexRoot/Figures}
\providecommand{\CodeDir}{\econtexRoot/Code}
\providecommand{\DataDir}{\econtexRoot/Data}
\providecommand{\SlideDir}{\econtexRoot/Slides}
\providecommand{\TableDir}{\econtexRoot/Tables}
\providecommand{\ApndxDir}{\econtexRoot/Appendices}

\providecommand{\ResourcesDir}{\econtexRoot/Resources}
\providecommand{\rootFromOut}{..} % APFach back to root directory from output-directory
\providecommand{\LaTeXGenerated}{\econtexRoot/LaTeX} % Put generated files in subdirectory
\providecommand{\econtexPaths}{\econtexRoot/Resources/econtexPaths}
\providecommand{\LaTeXInputs}{\econtexRoot/Resources/LaTeXInputs}
\providecommand{\LtxDir}{LaTeX/}
\providecommand{\EqDir}{\econtexRoot/Equations} % Put generated files in subdirectory

\providecommand{\local}{\LaTeXInputs/local}

\onlyinsubfile{% https://tex.stackexchange.com/questions/463699/proper-reference-numbers-with-subfiles
    \csname @ifpackageloaded\endcsname{xr-hyper}{%
      \externaldocument{\econtexRoot/SolvingMicroDSOPs}% xr-hyper in use; optional argument for url of main.pdf for hyperlinks
    }{%
      \externaldocument{\econtexRoot/SolvingMicroDSOPs}% xr in use
    }%
    \renewcommand\labelprefix{}%
    % Initialize the counters via the labels belonging to the main document:
}


\onlyinsubfile{\externaldocument{../SolvingMicroDSOPs}} % Get xrefs -- esp to appendix -- from main file; only works properly if main file has already been compiled;

\begin{document}
% \Habits{
\hypertarget{Multiple-State-Variables}{}
\section{Multiple State Variables}
We now wish to consider how the problem changes if there are multiple
state variables rather than just a single state variable.  The example
we will use will be the case where the utility from consumption
depends on the size of a `habit stock' which represents an average of
past levels of consumption.  Formally, the goal is to
\begin{equation}
  \max \left\{ \sum_{s=t}^{T} \DiscFac^{s-t} \uFunc({c}_{s},h_{s-1}) \right\}
\end{equation}
Now there are two state variables in the problem at time $t$, the
level of assets ${m}_{t}$ and the level of the habit stock $h_{t-1}$,
where the accumulation equation for ${m}_{t}$ is the same as before and
the transition equation for habits is
\begin{equation}\begin{gathered}\begin{aligned}
  h_{t}  & = h_{t-1} + \MPS({c}_{t}-h_{t-1}).
\end{aligned}\end{gathered}\end{equation}
That is, the habit stock at the end of this period is equal to the
habit stock at the end of last period plus a proportion of the gap
between the level of consumption chosen this period and the level of
the habit stock from last period.  In other words, habits `catch up'
to consumption at rate $\MPS$.

Assume that the utility function is given by
\begin{equation}
  \uFunc({c}_{t},h_{t-1}) = \frac{({c}_{t}/h_{t-1}^{\gamma})^{1-\CRRA}}{1-\CRRA}.
\end{equation}
Now that $u_{t}$ has two arguments we need to be able to distinguish
between the derivatives with respect to each argument.  Our notation
will be that the derivative of $u_{t}$ with respect to ${c}_{t}$ is
$u_{t}^{c}({c}_{t},h_{t-1})$ or $u_{t}^{c}$ for short, and analogously
for $u_{t}^{h}$.  Thus we have
\begin{equation}\begin{gathered}\begin{aligned}
  \uFunc^{c}_{t}  & = ({c}_{t} h_{t-1}^{-\gamma})^{-\CRRA}h_{t-1}^{-\gamma}  \\
                  & = {c}_{t}^{-\CRRA}h_{t-1}^{\CRRA\gamma-\gamma}  \\
  \uFunc^{h}_{t}  & = -\gamma ({c}_{t} h_{t-1}^{-\gamma})^{-\CRRA} {c}_{t} h_{t-1}^{-\gamma-1} \\
                  & = -\gamma {c}_{t}^{1-\CRRA}h_{t-1}^{\gamma\CRRA - \gamma -
                    1} = -\gamma ({c}_{t}/h_{t-1}) \uFunc^{c}_{t}.
\end{aligned}\end{gathered}\end{equation}

Bellman's equation for this problem (imposing liquidity constraints again) is
\begin{equation}\begin{gathered}\begin{aligned}
  {\vFunc}_{t}({m}_{t},h_{t-1})  & = \max_{\{{c}_{t}\}}  \{\uFunc({c}_{t},h_{t-1}) +
                                   \Ex_{t}[\DiscFac  {\vFunc}_{t+1}({m}_{t+1},h_{t}) ]\}
  \\ & \text{s.t.} \nonumber \\
  {m}_{t+1}  & = [{m}_{t}-{c}_{t}]\RNrm_{t+1} + \TranShkEmp_{t+1}\\
  h_{t}  & = h_{t-1} + \MPS ({c}_{t} - h_{t-1} ) \\
  {c}_{t} & \leq  {m}_{t} .
\end{aligned}\end{gathered}\end{equation}

As was done for utility above, define the derivatives of ${\vFunc}_{t}$ with
respect to each argument as ${\vFunc}_{t}^{m}({m}_{t},h_{t-1})$ (${\vFunc}_{t}^{m}$
for short) and ${\vFunc}_{t}^{h}$.  Also as above, we want to
define a function which corresponds to the expectation of the value of
ending period $t$ in a given position, but now the `position'
involves both the level of assets ${a}_{t}$ and the level of the habit
stock $h_{t}$:
\begin{equation}\begin{gathered}\begin{aligned}
  \vEndStge({a}_{t},h_{t})  & = \Ex_{t}[\DiscFac {v}_{t+1}(\RNrm_{t+1} {a}_{t}+{\TranShkEmp}_{t+1},h_{t}))]
\end{aligned}\end{gathered}\end{equation}
For future reference note that the derivatives are
\begin{equation}\begin{gathered}\begin{aligned}
  \vEndStge^{{a}}  & = \DiscFac \Ex_{t} [\RNrm_{t+1} {v}_{t+1}^{m}] \label{eq:vEndsdefn} \\
  \vEndStge^{h}  & = \DiscFac \Ex_{t} [{\vFunc}_{t+1}^{h}] 
\end{aligned}\end{gathered}\end{equation}
and the maximization problem can be rewritten
\begin{equation}\begin{gathered}\begin{aligned}
  {\vFunc}_{t}({m}_{t},h_{t-1})
  & =                                         \max_{\{{c}_{t}\}} %\{
    \uFunc({c}_{t},h_{t-1})  +  \DiscFac
    \vEndStge({m}_{t}-{c}_{t},h_{t-1}+\MPS({c}_{t}-h_{t-1}))
    % \}
  \\        & \text{s.t.} \nonumber
  \\  {c}_{t} & \leq  {m}_{t}.
\end{aligned}\end{gathered}\end{equation}

\hypertarget{The-Strategy}{}
\subsection{The Strategy}

This problem will be solved by a generalization of the strategy used
to solve the one-state problem.  Previously, with a period-$t+1$
consumption function in hand, we started by calculating
$\vFunc^{a}_{i}$ at the set of $n$ gridpoints ${a}_{i}$
contained in the variable \texttt{aVec}.  Now we will need to
have a grid of possible values for habits $h_{j}$ as well (in the
variable \texttt{hVec}) with, say, $m$ gridpoints.  We will then
calculate the values of $\vEndStge^{{a}}$ and $\vEndStge^{h}$
at all \textit{combinations} of the ${a}_{i}$ and $h_{j}$
gridpoints, for a total of $m \times n$ datapoints.  Finally, with
these results in hand, we can obtain the corresponding values of
consumption and construct the approximating interpolation to the
consumption function.

\hypertarget{Optimality-Conditions}{}
\subsection{Optimality Conditions}
\hypertarget{The-First-Orrder-Condition-for-c}{}
\subsubsection{The First Order Condition for ${c}_{t}$}
The FOC for this problem with respect to ${c}_{t}$ is:
\ifthenelse{\boolean{MyNotes}}{\marginpar{\tiny The $-\MPS
    \vEndStge^{h}_{t}$ term reflects the fact that if you consume a
    bit more today, tomorrow's habit stock will be larger by $\MPS$,
    which will make you worse off.}}{}
\begin{equation}\begin{gathered}\begin{aligned}
  0  & = \uFunc^{c}_{t} + \DiscFac \Ex_{t} \left[{\vFunc}^{m}_{t+1}(-\Rfree)+{\vFunc}^{h}_{t+1} \MPS\right]  \\
  \uFunc^{c}_{t}  & = \DiscFac \Ex_{t} [\RNrm_{t+1} {\vFunc}_{t+1}^{m}-\MPS {\vFunc}_{t+1}^{h}] \label{eq:ctfoc}
  \\            & = \left[\vEndStge^{{a}}-\MPS \vEndStge^{h} \right].
  % ({c}_{t}h_{t-1}^{-\gamma})^{-\CRRA}h_{t-1}^{-\gamma}  & = \DiscFac \Ex_{t} \left(\RNrm_{t+1} {\vFunc}^{m}_{t+1}-{\vFunc}^{h}_{t+1} \MPS\right) \label{eq:directfoc} \\
  % {c}_{t}^{-\CRRA}  & = h_{t-1}^{\gamma-\gamma\CRRA}\DiscFac \Ex_{t} \left(\RNrm_{t+1} {\vFunc}^{m}_{t+1}-{\vFunc}^{h}_{t+1} \MPS\right)
\end{aligned}\end{gathered}\end{equation}
Substituting the definition of $u_{t}^{c}$:
\begin{equation}\begin{gathered}\begin{aligned}
  {c}_{t}^{-\CRRA}h_{t-1}^{\CRRA\gamma-\gamma}  & = \left[\vEndStge^{{a}}-\MPS \vEndStge^{h} \right]
  \\  {c}_{t}                                     & = \left[h_{t-1}^{\gamma-\CRRA\gamma} \left(\vEndStge^{{a}}-\MPS \vEndStge^{h} \right) \right]^{-1/\CRRA} \label{eq:habitsfoc}.
\end{aligned}\end{gathered}\end{equation}
and the liquidity constraint implies that if the $\check{c}_{t}$ which
satisfies this equation is larger than ${m}_{t}$ the consumer spends
${m}_{t}$ rather than $\check{c}_{t}$.  The point at which the liquidity
constraint becomes binding is therefore implicitly defined by the equation
\begin{equation}\begin{gathered}\begin{aligned}
  {m}_{t}  & = \left[h_{t-1}^{\gamma-\CRRA\gamma} \left(\vEndStge^{{a}}(0,h_{t-1}+\MPS({m}_{t}-h_{t-1}))-\MPS \vEndStge^{h}(0,h_{t-1}+\MPS({m}_{t}-h_{t-1}))\right)  \right]^{-1/\CRRA}, \label{eq:LCbindsHabits}
\end{aligned}\end{gathered}\end{equation}
or incorporating the transition equation for
and $h_{t-1}$ that
\begin{equation}\begin{gathered}\begin{aligned}
  {c}_{t}  & = \left[h_{t-1}^{\gamma-\CRRA\gamma}
             \left(\vEndStge^{{a}}({m}_{t}-{c}_{t},h_{t-1}+\MPS({c}_{t}-h_{t-1}))
             -\MPS \vEndStge^{h}({m}_{t}-{c}_{t},h_{t-1}+\MPS({c}_{t}-h_{t-1})\right)
             \right]^{-1/\CRRA}. \nonumber
\end{aligned}\end{gathered}\end{equation}

This equation could be solved for a given grid of ${m}_{t}$ and
$h_{t-1}$ values to yield the two-dimensional matrix of values
necessary to construct an interpolating approximation to the
consumption function.  However, note that the value of ${m}_{t}$ at
which the constraint becomes binding depends on the level of
$h_{t-1}$.  This causes a problem, because the built-in {Mathematica}
interpolation algorithms require multidimensional interpolations to
contain a value for every possible combination of the independent
variables.  Thus, we would need to add each of the $m$ binding
points to the \texttt{mVec} variable, and to pair \textit{all} of
those bindingpoints to \textit{each} possible value in \texttt{hVec}.
This would immediately increase the number of points in the ${m}$
dimension by $m$, so the total number of gridpoints would now be
$(n+m) \times m$.  Thus this strategy would require increasing the
size of the matrix of interpolating values by a \textit{factor} of $m$
with a corresponding serious slowdown in computational speed.

Fortunately, there is a way to get around this problem.  To
understand it, start by noticing the translation in this context of
the trick used to solve the one-dimensional consumption problem.
Recall that there the trick was to start with the grid of values in
\texttt{aVec} and find the unique ${\cNrm}_{i}$, and
consequently ${m}_{i}={a}_{i}+{\cNrm}_{i}$, associated with
each ${a}_{i}$ thus constructing the set of ${\cNrm}_{i}$ and
${m}_{i}$ values without solving a maximization problem or
numerically finding a root.  Here, the most effective strategy is to
use \texttt{aVec} to define a set of ${a}_{t}$ values but use
\texttt{hVec} to define a set of values for $h_{t-1}$, with the
transition equation determining the $h_{t}$ to be paired with the
given ${a}_{t}$.  That is, defining the set $\mathcal{L} =
\{\{{a}_{1},h_{1}\},\{{a}_{1},h_{2}\},\ldots,\{{a}_{m},h_{n_{\TranShkEmp}}\}\}$,
we find the set of values of ${\cNrm}_{i}$ that satisfy
\begin{equation}\begin{gathered}\begin{aligned}
  {\cNrm}_{k}^{\ell}  & = \left[(h_{k}^{\ell})^{\gamma-\CRRA\gamma}
                        \left(\vEndStge^{{a}}({a}_{k}^{\ell},h_{k}^{\ell}
                        +\MPS({\cNrm}_{k}^{\ell}-h_{k}^{\ell}))
                        -\MPS
                        \vEndStge^{h}({a}_{k}^{\ell},h_{k}^{\ell}+\MPS({\cNrm}_{k}^{\ell}-h_{k}^{\ell})\right)
                        \right]^{-1/\CRRA}. \nonumber
\end{aligned}\end{gathered}\end{equation}
and the budget constraint again gives us the ${m}_{k}^{\ell}$
associated with the given $\{{a}_{t},h_{t-1}\}$ pair.

However, there is a problem: {Mathematica}'s built-in two-dimensional
interpolation routines require a complete set of values of ${c}_{t}$
for \textit{each distinct} $\{{m}_{k},h_{j}\}$ pair.  But the list of
${m}_{k}$ gridpoints associated with each distinct point in
\texttt{hVec} will be different.  Thus to use the built-in routines
it would be necessary to take the union of all the ${m}_{k}$
produced and then construct values of ${\cNrm}_{k}$ for each of
these.  If there are $m$ values in \texttt{aVec}, this would
mean solving the problem at a total of $m \times m \times n$
girdpoints, only $m \times n$ of which would have been produced
during the first-round procedure.

There is a a much better solution: Manual interpolation.  This works
as follows.  For each distinct value of $h_{t,i}$ the value of
${\cNrm}_{t,i}$ associated with each given ${a}_{t,i}$ is
calculated as before, starting with ${a}_{1,t}=0$ which yields
the value of ${m}_{t,i}$ and ${\cNrm}_{t,i}$ at which the
constraint begins to bind for this combination of ${a}_{t,i}$
and $h_{t,i}$.  Then an \texttt{InterpolatingFunction} is
constructed as before.

Thus we will have a collection of $j$ interpolating consumption
functions, one for each value in \texttt{hVec}.  Denote these
functions by $\Alt{\Hi{\cFunc}}_{t}({m}_{t},j)$.

Now consider how to estimate $\cFunc_{t}({m}_{t},h_{t})$ for arbitrary
${m}_{t}$ and $h_{t}$.  Define $\underline{h}_{t}$ as the value, and
$\underline{j}_{t}$ as the index of the nearest \texttt{hVec} point
below $h_{t}$ and define $\bar{h}_{t}$ analogously for the nearest
gridpoint above $h_{t}$.  Then we define our approximation to the
consumption function as
\begin{displaymath}
  \Alt{\Hi{\cFunc}}_{t}({m}_{t},h_{t}) =
  \Alt{\Hi{\cFunc}}({m}_{t},\underline{j})+\left(\frac{h_{t}-\underline{h}_{t}}
    {\bar{h}_{t}-\underline{h}_{t}}\right)
  \left(\Alt{\Hi{\cFunc}}({m}_{t},\overline{j})
    -\Alt{\Hi{\cFunc}}({m}_{t},\underline{j})\right)
\end{displaymath}
with points outside the \texttt{hVec} points constructed by
extrapolation.

Note that a liquidity constraint is again incorporated at almost
zero cost: Simply make sure that the lowest value in
\texttt{aVec} is 0, and append the point $\{0.,0.\}$ as the
bottommost point in each of the lists of $\{{m}_{t},{c}_{t}\}$
assoicated with the various values in \texttt{hVec}.

\hypertarget{Applying-the-Envelope-Theorem}{}
\subsubsection{Applying the Envelope Theorem}
The Envelope theorem on ${m}_{t}$ says:
\begin{equation}\begin{gathered}\begin{aligned}
  {\vFunc}^{m}_{t}  & = \frac{\partial {\vFunc}_{t}}{\partial {m}_{t}} + \overbrace{\frac{\partial {\vFunc}_{t}}{\partial {c}_{t}}}^{=0}\frac{\partial {c}_{t}}{\partial {m}_{t}} \label{eq:vxt} \\ % old \label{eq:envelopex}
  {\vFunc}_{t}^{m}  & = \DiscFac  \Ex_{t} [\RNrm_{t+1} {\vFunc}_{t+1}^{m}] 
\end{aligned}\end{gathered}\end{equation}

Substituting this into the FOC equation~(\ref{eq:ctfoc}) gives
\begin{equation}\begin{gathered}\begin{aligned}
  \uFunc^{c}_{t}  & = {\vFunc}_{t}^{m}-\Ex_{t}[\DiscFac \MPS {\vFunc}_{t+1}^{h}] % old \label{eq:ctfocsubx}
  \\ {\vFunc}_{t}^{m}  & = \uFunc^{c}_{t}+\Ex_{t}[\DiscFac \MPS {\vFunc}_{t+1}^{h}] \label{eq:vtfoc}
  \\            & = \uFunc^{c}_{t}+\MPS \vEndStge^{h} % old \label{eq:vxtformula}.
\end{aligned}\end{gathered}\end{equation}

What if the consumer is liquidity constrained?  It is useful here to \ifthenelse{\boolean{MyNotes}}{\marginpar{\tiny By `is constrained' I mean the constraint is binding right now.}}{}
rewrite Bellman's equation:
\begin{equation}\begin{gathered}\begin{aligned}
  {\vFunc}_{t}({m}_{t},h_{t-1})  & = \uFunc({c}_{t},h_{t-1}) +  \DiscFac \Ex_{t}
                                   \left[{\vFunc}_{t+1}(({m}_{t}-{c}_{t})\RNrm_{t+1}+{\TranShkEmp}_{t+1},h_{t-1}+\MPS({c}_{t}-h_{t-1}))
                                   \right] \nonumber
\end{aligned}\end{gathered}\end{equation}
Substituting in the fact that ${c}_{t}={m}_{t}$ (because the consumer is constrained)
\begin{equation}\begin{gathered}\begin{aligned}
  {\vFunc}_{t}({m}_{t},h_{t-1})  & = \uFunc({c}_{t},h_{t-1}) +  \Ex_{t}\left[\DiscFac
                                   {\vFunc}_{t+1}({\TranShkEmp}_{t+1},h_{t-1}+\MPS({c}_{t}-h_{t-1}))\right]
                                   \nonumber
\end{aligned}\end{gathered}\end{equation}
Thus $\partial {\vFunc}_{t}/\partial {m}_{t} = 0$, and because the liquidity
constraint implies that $\partial {c}_{t}/\partial {m}_{t} = 1$,
equation (\ref{eq:vxt}) becomes
\begin{equation}\begin{gathered}\begin{aligned}
  {\vFunc}_{t}^{m}   & =  \frac{\partial {\vFunc}_{t}}{\partial {c}_{t}}
  \\              & = \uFunc_{t}^{c} + \DiscFac \Ex_{t} [\MPS {\vFunc}_{t+1}^{h}]
\end{aligned}\end{gathered}\end{equation}
which is identical to the expression (\ref{eq:vtfoc}) for ${\vFunc}_{t}^{m}$
for the unconstrained consumer.

The Envelope theorem for $h_{t-1}$ says:
\begin{equation}\begin{gathered}\begin{aligned}
  {\vFunc}_{t}^{h}  & = \frac{\partial {\vFunc}_{t}}{\partial h_{t-1}} + \overbrace{\frac{\partial {\vFunc}_{t}}{\partial {c}_{t}}}^{=0}\frac{\partial {c}_{t}}{\partial h_{t-1}} \nonumber
  \\  & = \uFunc^{h}_{t} + \DiscFac \Ex_{t} \left[{\vFunc}_{t+1}^{h}\frac{\partial h_{t}}{\partial h_{t-1}} \right]\nonumber
  \\  & = \uFunc^{h}_{t} + \DiscFac \Ex_{t} [(1-\MPS) {\vFunc}_{t+1}^{h}] \nonumber
  \\  & = \uFunc^{h}_{t} + (1-\MPS) \vEndStge^{h} \label{eq:vhtformula}
\end{aligned}\end{gathered}\end{equation}
% & = -\gamma {c}_{t}^{1-\CRRA} h_{t-1}^{\gamma \CRRA - \gamma -1} + (1-\MPS) \DiscFac \Ex_{t} {\vFunc}_{t+1}^{h} \label{eq:vhtfoc}
What if the consumer is constrained?  In that case while $\partial
{\vFunc}_{t}/\partial {c}_{t} \neq 0$, $\partial {c}_{t}/\partial h_{t-1} = 0$, so as
with ${\vFunc}_{t}^{m}$ the constraint has no effect on the expression for ${\vFunc}^{h}_{t}$.

\hypertarget{Transforamtions}{}
\subsection{Transformations}

In the time-separable consumption problem, there was a unique
transformation that allowed us to unwind much of the linearity
attributable to the curvature of the utility function; this was
possible because that problem has an analytical solution in the
perfect foresight case, and because the problem is one with a single
state variable.

In the habits model, neither of these conditions is true, and
consequently the appropriate transformation strategy is not obvious.
To see this, consider the last period of life and suppose there is no
uncertainty.  Then ${\vFunc}_{\dcsn(T)}^{m} = \uFunc_{T}^{c}$ will be of the form
${c}_{T}^{-\CRRA}h_{T-1}^{\CRRA \gamma - \gamma}$, while ${\vFunc}_{\dcsn(T)}^{h} =
\uFunc^{h}$ will be something of the form $-\gamma
{c}_{T}^{1-\CRRA}h_{T-1}^{\gamma\CRRA-\gamma-1}$.  It is possible to
exponentiate either of these equations to make it linear in one or the
other of ${c}$ or $h$, but not both.

However, there is a two-step procedure that solves the problem.  In
the first step, the object in question is transformed as in the
original problem without habits.  For example, the transformed utility
function for the last period of life would be
\begin{equation}\begin{gathered}\begin{aligned}
  n_{T}^{c}  & = \left[\uFunc^{c}_{T}\right]^{-1/\CRRA}
  \\        & = {c}_{T}h_{T-1}^{\gamma(1/\CRRA-1)}.
\end{aligned}\end{gathered}\end{equation}

This leaves us with an equation that is linear in ${c}_{T}$ but
nonlinear in $h_{T-1}$.  So now we do a second transformation:
dividing the $n_{T}$ function by $h_{T-1}^{\gamma(1/\CRRA-1)}$.  This
yields a function $\hat{n_{\TranShkEmp}}_{T}^{c}$ that can be transformed back into
$\uFunc^{c}_{T}$ on demand, but is itself linear in ${c}_{t}.$  Analogously,
in the last period of life the marginal value of habits is given by
\begin{equation*}\begin{gathered}\begin{aligned}
  {\vFunc}_{\dcsn(T)}^{h}  & = -\gamma {c}_{T}^{1-\CRRA}h_{T-1}^{\gamma(\CRRA-1)-1}
\end{aligned}\end{gathered}\end{equation*}
so the natural normalization procedure is to define
\begin{equation}\begin{gathered}\begin{aligned}
  \Lambda_{t}^{h}
  & =                           \left[\frac{h_{t-1}^{1-\gamma(\CRRA-1)}{\vFunc}_{t}^{h}}{-\gamma}\right]^{1/(1-\CRRA)}
\end{aligned}\end{gathered}\end{equation}
and similarly for $\vEndStge^{h}$.


\hypertarget{Transforming-cFunc}{}
\subsection{Transforming $\cFunc(\bullet,h_{t-1})$}

The habits problem presents an additional difficulty: the consumption
function is highly nonlinear in $h_{t-1}$ for a given ${m}_{t}$.   To
see this, start with the first order condition for the problem in the
second-to-last period of life (setting $\DiscFac = \RNrm_{t+1} = 1$ and assuming
away uncertainty for transparency):
\begin{equation}\begin{gathered}\begin{aligned}
  \uFunc_{t}^{c}    & = \uFunc_{t+1}^{c} - \MPS {\vFunc}_{t+1}^{h}
  \\      \uFunc_{T-1}^{c}  & = \uFunc_{T}^{c} - \MPS \uFunc_{T}^{h}
  \\   & =  \uFunc_{T}^{c}\left(1 + \gamma \MPS
         ({c}_{T}/h_{T-1})\right)
  \\      {c}_{T-1}^{-\CRRA}
                    & =                                      {c}_{T}^{-\CRRA}\left(\frac{h_{T-1}}{h_{T-2}}\right)^{\gamma(\CRRA-1)}\left(1 + \gamma \MPS
                      ({c}_{T}/h_{T-1})\right)  \\
  {c}_{T-1}
                    & =                     ({m}_{T-1}-{c}_{T-1})\left(\frac{h_{T-1}}{h_{T-2}}\right)^{\gamma(1/\CRRA-1)}\left(1 +
                      \gamma \MPS \left[\frac{{m}_{T-1}-{c}_{T-1}}{(1-\MPS)h_{T-2}+\MPS
                      {c}_{T-1}}\right]\right)^{-1/\CRRA}
  \\      1
                    & =                     \left(\frac{{m}_{T-1}-{c}_{T-1}}{{c}_{T-1}}\right)\left(\frac{(1-\MPS)h_{T-2}+\MPS {c}_{T-1}}{h_{T-2}}\right)^{\gamma(1/\CRRA-1)}\left(1 +
                      \gamma \MPS \left[\frac{{m}_{T-1}-{c}_{T-1}}{(1-\MPS)h_{T-2}+\MPS
                      {c}_{T-1}}\right]\right)^{-1/\CRRA} \nonumber
  \\ \label{eq:oneeq}
\end{aligned}\end{gathered}\end{equation}

Now conjecture that for fixed ${m}_{T-1}$
\begin{equation}
  \lim_{h_{T-1} \rightarrow 0} {c}_{T-1} = \mu h_{T-2}^{m}
  \label{eq:mueq}
\end{equation}
for some $\mu<1, \mu>1$.  Then the limit of \eqref{eq:oneeq} as $h_{T-2}
\rightarrow \infty$ is given by
\begin{equation*}\begin{gathered}\begin{aligned}
  1  & = ({m}_{T-1}/{c}_{T-1})\left(\frac{\MPS
       {c}_{T-1}}{h_{T-2}}\right)^{\gamma(1/\CRRA-1)}\left(\gamma
       \left[\frac{{m}_{T-1}}{{c}_{T-1}}\right]\right)^{-1/\CRRA}
  \\
     & =                 {c}_{T-1}^{\gamma(1/\CRRA-1)-1+1/\CRRA}h_{T-2}^{-\gamma(1/\CRRA-1)}\MPS^{\gamma(1/\CRRA-1)}(\gamma {m}_{T-1})^{-1/\CRRA}
  \\
     & =                 {c}_{T-1}^{(1+\gamma)(1/\CRRA-1)}h_{T-2}^{-\gamma(1/\CRRA-1)}\MPS^{\gamma(1/\CRRA-1)}(\gamma {m}_{T-1})^{-1/\CRRA}
  \\
     & =                 {c}_{T-1}^{(1+\gamma)}h_{T-2}^{-\gamma}\MPS^{\gamma}(\gamma {m}_{T-1})^{1/(\CRRA-1)}
  \\ {c}_{T-1}    & = h_{T-2}^{\gamma/(1+\gamma)}\MPS^{-\gamma}(\gamma
                    {m}_{T-1})^{\frac{1}{(\gamma+1)(\CRRA-1)}}
\end{aligned}\end{gathered}\end{equation*}
which confirms the conjecture \eqref{eq:mueq}.  On the other hand,
suppose that we postulate that
\begin{equation}
  \lim_{h_{T-2} \rightarrow \infty} {c}_{T-1} = \kappa.
  \label{eq:kappaguess}
\end{equation}

Under this conjecture, the limit of the RHS of \eqref{eq:oneeq} as
$h_{T-2} \rightarrow \infty$ is
\begin{equation*}\begin{gathered}\begin{aligned}
  \kappa
  & =                  (1-\kappa)\left(1-\MPS\right)^{\gamma(1/\CRRA-1)}
  \\      \kappa\left(1+(1-\MPS)^{\gamma(1/\CRRA-1)}\right)  & = \left(1-\MPS\right)^{\gamma(1/\CRRA-1)}
  \\      \kappa
  & =                          \left(\frac{\left(1-\MPS\right)^{\gamma(1/\CRRA-1)}}
    {1+(1-\MPS)^{\gamma(1/\CRRA-1)}}\right)
\end{aligned}\end{gathered}\end{equation*}
confirming conjecture \eqref{eq:kappaguess}.

This combination of results indicates that the consumption function
must be globally strongly nonlinear in $h_{T-1}$ (a nonlinearity which
also arises in successive earlier periods $T-2$ and so on).

If we wish to normalize the consumption rule by something that will
help to make it approximately linear, that normalizing function will
need approach proportionality to $h_{t-1}^{\gamma/(1+\gamma)}$ as
$h_{t-1}$ goes to zero but approach a constant as $h_{t-1}$ approaches
infinity.  The function
\begin{equation}
  n(h_{t-1}) =
  h_{t-1}^{\gamma/(1+\gamma)}(h_{t-1}+\mu)^{-\gamma/(1+\gamma)}
\end{equation}
has precisely these characteristics for $\mu>0$.  Some
experimentation with possible values of $\mu$ led to the conclusion
that under the baseline parameter values this function does a good job
of linearizing the relationship between consumption and $h_{t-1}$ for
a value of $\mu = 0.04$.

The combined transformations for the various variables are thus
\begin{equation}\begin{gathered}\begin{aligned}
  \cFunc_{\overline{t}}^{{a}}({a}_{t},h_{t})        & = [\vEndStge^{{a}}({a}_{t},h_{t})]^{-1/\CRRA} h_{t}^{\gamma(1-1/\CRRA)} %\label{eq:vEndstransform} \\
  \Lambda_{t}^{m}({m}_{t},h_{t-1})   & = [{\vFunc}_{t}^{m}({m}_{t},h_{t-1})]^{-1/\CRRA} h_{t-1}^{\gamma(1-1/\CRRA)} % old \label{eq:vxtransform} \\
  \cFunc_{\overline{t}}^{h}({a}_{t},h_{t})
                                                    & =                                                  [-h_{t}^{1-\gamma(\CRRA-1)}\vEndStge^{h}({a}_{t},h_{t})/\gamma]^{1/(1-\CRRA)}  \label{eq:vEndhtransform}
  \\      \Lambda_{t}^{h}({m}_{t},h_{t-1})   & = [-h_{t-1}^{1-\gamma(\CRRA-1)}{\vFunc}_{t}^{h}({m}_{t},h_{t-1})/\gamma]^{1/(1-\CRRA)} % \old \label{eq:vhtransform}
  \\      \Lambda_{t}({m}_{t},h_{t-1})       & = [(1-\CRRA)
                                               {\vFunc}_{t}({m}_{t},h_{t-1})]^{1/(1-\CRRA)} h_{t-1}^{\gamma} % old \label{eq:vtransform}
  \\  \chi({m}_{t},h_{t-1})       & = \cFunc({m}_{t},h_{t-1})h_{t-1}^{-\gamma/(1+\gamma)}(h_{t-1}+\mu)^{\gamma/(1+\gamma)}.
\end{aligned}\end{gathered}\end{equation}
where these are the functions that are actually approximated by
interpolation and the objects of interest (like the consumption
function) are obtained by reversing the transformations.

\hypertarget{The-Program}{}
\subsection{The Program}
The consumption problem with habit formation is solved in
\texttt{habits.m}, whose structure closely follows that of
\texttt{multiperiod.m}.

Assuming the problem has been solved up to period $t+1$ (and thus we
have numerical functions $\hat{\vFunc}_{t+1}^{m}({m}_{t+1},h_{t})$ and
$\hat{\vFunc}_{t+1}^{h}({m}_{t+1},h_{t})$),
\begin{enumerate}

\item Form a list called \texttt{ArgArray} of all possible
  combinations of the values in \texttt{aVec} and
  \texttt{hVec}, and index the components of that list by $k$.  Thus
  if there are $m$ points in both grids we have \texttt{ArgArray}=
  $\{\{{a}_{1},h_{1}\}$, $\{{a}_{1},h_{2}\},\ldots$,
  $\{{a}_{1},h_{m}\}$, $\{{a}_{2},h_{1}\}$,
  $\{{a}_{2},h_{2}\},\ldots$, $\{{a}_{2},h_{m}\},
  \{{a}_{m},h_{1}\}$, $\{{a}_{m},h_{2}\},\ldots$,
  $\{{a}_{m},h_{m}\}\}$.  Designate this list as $\mathcal{L}$
  with individual members $\{\ell_{1}, \ell_{2}, \ldots \ell_{m \times
    m}\}$.  Finally, define the notation $\bullet^{\ell}_{k}$ to mean
  ``the value of $\bullet$ associated with the $k$th element of the
  list $\mathcal{L}$; e.g. $h^{\ell}_{1} = h_{1}$, $h^{\ell}_{2} =
  h_{2}$, and ${a}^{\ell}_{2}={a}_{1}$.''

  Now at each of the $k$ locations in \texttt{ArgArray} calculate the
  value of $\cFunc_{\overline{t}}^{{a}}$ and $\cFunc_{\overline{t}}^{h}$ (from
  equations (\ref{eq:vEndsdefn}) and (\ref{eq:vEndhdefn})) and
  \eqref{eq:vEndstransform} and \eqref{eq:vEndhtransform},
  \begin{equation}\begin{gathered}\begin{aligned}
    \cFunc_{k,t}^{{a}}  & = (h_{k}^{\ell})^{\gamma(1-1/\CRRA)}
                             \left(\DiscFac \Ex_{t}
                             \left[
                             \RNrm_{t+1} (\hat{\vFunc}_{t+1}^{m}(\RNrm_{t+1} s^{\ell}_{k} +
                             {\TranShkEmp}_{t+1},h^{\ell}_{k}))
                             \right]
                             \right)^{-1/\CRRA}
    \\      \cFunc_{k,t}^{h}
                           & =                                         \left(-(h_{k}^{\ell})^{1-\gamma(\CRRA-1)} \DiscFac \Ex_{t}
                             \left[
                             \hat{\vFunc}_{t+1}^{h}(\RNrm_{t+1} s^{\ell}_{k} +
                             {\TranShkEmp}_{t+1},h^{\ell}_{k})
                             \right]/\gamma
                             \right)^{1/(1-\CRRA)},
  \end{aligned}\end{gathered}\end{equation}
generating lists of values
$\cFunc^{a}_{k,t}$,$\cFunc^{h}_{k,t}$.

\item Construct interpolating functions $\hat{\cFunc}^{a}_{t}({a}_{t},h_{t})$
  and $\hat{\cFunc }^{h}_{t}({a}_{t},h_{t})$, from which we can
  obtain $\hat{\vFunc}_{\EndStge}^{a}$ and $\hat{\vFunc}_{\EndStge}^{h}$
  via the inverse of the transformations (\ref{eq:vEndstransform})
  and (\ref{eq:vEndhtransform}).

\item Loop over the $m$ values of $h$ in \texttt{hVec}, indexing them
  by $j$; for each $j$:
  \begin{itemize}
  \item Loop over ${a}_{i}$ finding the optimal ${c}_{a}$ associated
    with this ${a}_{i}$ and $h_{j}$ from the formula in equation
    \eqref{eq:habitsfoc}:
    \begin{equation}
      {\cNrm}_{i} = \left[h_{j}^{\gamma-\CRRA \gamma}
        (\vEndStge^{{a}}({a}_{i},(1-\MPS) h_{j}+ {\cNrm}_{i})
        -\MPS \vEndStge^{h}({a}_{i},(1-\MPS)
        h_{j}+{\cNrm}_{i}))\right]^{-1/\CRRA} \nonumber
    \end{equation}

  \item Construct ${m}_{i} = {\cNrm}_{i}+{a}_{i}$

  \item Construct \texttt{ct[mt,j]} as a linear interpolation of the
    $\{{m}_{i},{\cNrm}_{i}\}$
  \end{itemize}
\end{enumerate}

With $\Alt{\Hi{\cFunc}}_{t}({m}_{t},h_{t-1})$, $\hat{\vFunc}_{\EndStge}^{{a}}$
and $\hat{\vFunc}_{\EndStge}^{h}$ in hand we can obtain ${\vFunc}^{m}_{t}$
and ${\vFunc}^{h}_{t}$ from \eqref{eq:vtfoc} and
\eqref{eq:vhtformula}. Thus we have generated $\Alt{\Hi{\cFunc}}_{t}$,
$\hat{\vFunc}^{m}_{t}$ and $\hat{\vFunc}^{h}_{t}$ from $\hat{\vFunc}^{a}_{t+1}$ and
$\hat{\vFunc}^{h}_{t+1}$, and we can continue the iteration indefinitely.

The problem is solved in the program \texttt{habits.m}.  Details of
the {Mathematica}~implementation follow those described above for
\texttt{multiperiod.m} closely, and so need not be described here.
The program generates a three-D figure showing the consumption rule
$\cFunc_{t}({m}_{t},h_{t-1})$ for the first period of `life.'  The figure
behaves as one would expect: consumption is increasing in the level
of resources and in the level of the habit stock.\footnote{For a
  detailed analysis of some of the properties of this habit formation
  model, see Carroll~\citeyearpar{carroll:RiskyHabits}}


% }{}
\end{document}
