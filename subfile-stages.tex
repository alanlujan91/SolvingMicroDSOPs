  Although our focus so far has been on the consumer's consumption problem once $\mNrm$ is known, for constructing the computational solution it will be useful to distinguish a sequence of three {\moves} (we use the word `\moves' to capture the fact that we are not really thinking of these as occurring at different moments in time -- only that we are putting the things that happen within a given moment into an ordered sequence).  The first {\move} captures calculations that need to be performed before the shocks are realized, the middle {\move} is after shocks have been realized but before the consumption decision has been made (this corresponds to the timing of $\vFunc$ in the treatment above), and the final {\move} captures the situation \textit{after} the consumption decision has been made.
